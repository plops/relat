\documentclass{scrartcl}
\usepackage{amsmath}
\usepackage{units}
\begin{document}
\title{Meinel: Relativistische Physik}
\maketitle
% 44 leute
% seminar freitag 14:15-16 seminar 102 abbeanum
% erster seminar termin ist vorlesung

% klausur zum schluss, zulassung zur klausur nur bei regelmaessiger
% teilnahme an uebungen, aufgaben werden nicht kontrolliert in der
% uebung von uns vorgestellt, stephan valenta
\section{Literatur}
Drei Buecher an die er sich besonders stark anlehnt

\begin{itemize}
\item J.B.~Hartle ``Gravity'', Add.~Wesley 2003
\item Landau/Lifschitz, Band~II
\item H. Stephani, ``Relativity'', Cambridge Univ.\ Press 2004
\end{itemize}

\section{Einfuehrung in die spezielle Relativitaetstheorie}
\subsection{Das Relativitaetsprinzip der klassischen Mechanik}
\begin{quote}
  Fuehrt man identische experimente in unterschiedlichen
  Inertialsystemen durch, erhaelt man identische Resultate.
\end{quote}

Uebergang von einem Inertialsystem $\Sigma(x,y,z)$ zu einem anderen
$\Sigma'(x',y',z')$:
Koordinatentransformation

\paragraph{Beispiele:}
\begin{enumerate}
\item Verschiebung entlang der x-Achse.
  \begin{align}
    x'&=x-d, \quad d=\textrm{const.}\\
    y'&=y\\
    z'&=z\\
    (t'&=t)
\end{align}



\begin{verbatim}
 | y         |  y'
 |           |
 |           |
 |           |
 |           |
 |    d      |
 +---------->+
 |           |
 |           |
 |           |
 |           |
 |           |
 +-----------+----------------------------------------      x, x'
\end{verbatim}

\item Drehung  um die z-Achse


\begin{verbatim}
                           |
            \              |
             \-            |
               \           |                            /-   x'
                \-         |                        /---
                  \        |                    /---
                   \-      |                /---
                     \     |             /--
                      \-   |         /---
                        \  |     /---
                         \-| /---          phi
 ------------------------/-\----------------------------------------------- x
                      /--  | \
                  /---     |  \-
              /---         |    \
          /---             |     \-
        --                 |       \
                           |        \-
                           |          \
                           |           \-
                                         \
\end{verbatim}

\begin{align}
  x'&=\cos(\phi) x + \sin(\phi) y, \quad \phi=\textrm{const.} \\
  y'&=-\sin(\phi) x + \cos(\phi) y \\
  z'&=z \\
  (t'&=t)
\end{align}

\item Gleichfoermige Bewegung entland der x-Achse

\begin{align}
  x'&=x-vt,\quad v=\textrm{const.}\\
  y'&=y \\
  z'&=z \\
  t'&=t
\end{align}
Galileitransformation

\begin{verbatim}
  | y         |  y'
  |           |
  |           |
  |           |
  |           |
  |    vt     |
  +---------->+
  |           |
  |           |
  |           |
  |           |
  |           |
  +-----------+----------------------------------------      x, x'
\end{verbatim}


\end{enumerate}

(1) und (2) sind rein raeumliche Transformationen aber auch bei (3)
wird die Zeit nicht transformiert (``absolute Zeit'')!

Fuer ein kraeftefreies Teilchen gilt gemaess 1. Axiom:
\begin{align}
  x_{,t^2}&=0, y_{,t^2}=0, y_{,t^2}=0 \\
  (d,\phi,v)=\textrm{const.}\\
  \rightarrow
  x'_{,t^2}&=0, y'_{,t^2}=0, y'_{,t^2}=0
\end{align}

\renewcommand{\d}{\textrm{d}}

\paragraph{Bemerkung}
Die zugrundeliegende Geometrie des euklidischen Raums wird in
kartesischen Koordinaten durch das ``Linienelement'' $\d S^2=(\d
S)^2=\d x^2+ \d y^2 + \d z^2$ charakterisiert (Satz des Pythagoras).

Die Form $\d S^2$ ist invariant bei Verschiebungen und Drehungen:
\begin{align}
  \d S^2=\d x^2+ \d y^2 + \d z^2=\d x'^2+ \d y'^2 + \d z'^2
\end{align}
Aufgrund der Symmetrieeigenschaften des euklidischen Raums.

\paragraph{Beweis}
(1) trivial

(2)
\begin{align}
  \d x'&=\cos(\phi) \d x + \sin(\phi) \d y \\
  \d y'&=-\sin(\phi) \d x + \cos(\phi) \d y \\
  \rightarrow \d x'^2 + \d y'^2 = \d x^2 + \d y^2
\end{align}

(3) ist fuer fixiertes $t$ auch eine Verschiebung

\subsection*{Addition von Geschwindigkeiten}
Fuer eine Galilei-Transformation gilt:

\begin{align}
  V'&=x'_{,t'} = x'_{,t} = V = x_{,t} - v \\
  y'_{,t'} &= y'_{,t} \\
  z'_{,t'} &= z'_{,t}
\end{align}

Ein Teilchen, das sich im System $\Sigma'$ mit einer Geschwindigkeit
$V'$ in $x'-$Richtung bewegt, bewegt sich also im System $\Sigma$ mit
einer Geschwindigkeit
\begin{align}
\label{eqn:addition-geschwindigkeiten}
  V = V' + v
\end{align}
in $x-$Richtung




\begin{verbatim}
              +------------------------------+
              |                              |
              |                              |
              |    o  --->  V'               |   ----> v
              |                              |
              |                              |
              |                              |
              +------------------------------+
                    o                    o
 ------------------------------------------------------------
\end{verbatim}

\section{Konstanz der Lichtgeschwindigkeit}

\paragraph{Frage:}
Gilt das Relativitaetsprinzip (R) auch fuer nicht-mechanische Vorg\"ange,
z.\ B.\ elektromagnetische Erscheinungen inklusive Lichtausbreitung?

\paragraph{Experimenteller Befund:}
Die Vakuum-Lichtgeschwindigkeit hat in allen Inertialsystemen den
exakten Wert $c=\unit[299792.458]{\textrm{km}/\textrm{s}}$.

Versuch von Michelson \& Morley 1887 (unabhaengig von der
Geschwindigkeit der Lichtquelle)

\paragraph{Problem:}
\ref{eqn:addition-geschwindigkeiten}


\paragraph{Antwort:} Ja!

\paragraph{Konsequenz:} Galilei-Transformation und damit klassische (=
Newtonsche) Mechanik sind falsch!!  (Genauer: sind nur
n\"aherungsweise g\"ultig f\"ur Geschwindigkeiten $<< c$)

$\rightarrow$ spezielle Relativitaetstheorie (Einstein 1905)

\paragraph{Bemerkung:}
Die Vakuum-Maxwell-Gleichungen der Elektrodynamik gelten in jedem
Inertialsystem und liefern den Wert $c$ fuer die Lichtgeschwindigkeit!
Sie sind nicht galilei- sondern lorentzinvariant,
vgl. spaeter\footnote{Man vermutete dass Maxwellgleichungen nur im
  System gelten, wo der \"Ather ruht. Das wurde geklaert. Die moderne
  Kosmologie hat aber fast wieder aehnliche Z\"uge (Stichwort dunkle
  Energie).}.

\subsection{Relativitaet der Gleichzeitigkeit}


Wagen ruht in $\Sigma'$

Von $A$ und $B$ wird je ein Lichtsignal in Richtung M (Mitte)
gesendet. Beide Lichtsignale kommen gleichzeitig bei M an!

$\rightarrow$ Beide Signale werden gleichzeitig (in $\Sigma'$) von $A$
und $B$ ausgesendet.



\begin{verbatim}
  ^   A              M              B
t'|   +--------------+--------------+
      |              |              |
      |             *|*             |
      |              |              |
      +--------------+--------------+


      +--------------+--------------+
      |              |              |
      |      *       |       *      |
      |              |              |
      +--------------+--------------+


      +--------------+--------------+
      |              |              |
      |*             |             *|
      |              |              |
      +--------------+--------------+
\end{verbatim}


Beurteilung vom System $\Sigma$ aus (Wagen bewegt sich nach rechts).

\begin{verbatim}
  ^             A              M              B
t |             +--------------+--------------+
                |              |              |
                |             *|*             |
                |              |              |
                +--------------+--------------+


           +--------------+--------------+
           |              |              |
           |      *       |             *|
           |              |              |
           +--------------+--------------+


      +--------------+--------------+
      |              |              |
      |*             |              |
      |              |              |
      +--------------+--------------+
\end{verbatim}


Damit die Signale gleichzeitig bei $M$ ankommen, muss das Signal bei
$A$ eher abgesendet werden.

\paragraph{Also:} Gleichzeitigkeit an unterschiedlichen Orten haengt
vom Bezugssystem ab!

\begin{description}
\item[$\Sigma$:] erst wird das Signal bei $A$, sp\"ater das bei $B$
  ausgesendet
\item[$\Sigma'$:] Die Signale $A$ und $B$ werden gleichzeitig
  ausgesendet.
\end{description}

$\rightarrow$ Das Konzept der absoluten Zeit muss aufgegeben werden!

Raum \& Zeit werden kombiniert zur Raumzeit

\subsection{Die Raumzeit}

\paragraph{Ereignis} Charakterisiert durch Ort, an dem es stattfindet,
und durch den Zeitpunkt, zu dem es geschieht. = Punkt in
vierdimensionalen Raum (``Raumzeit'')

Koordinaten eines Inertialsystems $\Sigma$: $x,y,z,ct$

$t$ ist die Zeit, die eine in $\Sigma$ ruhende Uhr
($x=\textrm{const.},y=\textrm{const.},z=\textrm{const.}$) anzeigt!

[All in $\Sigma$ ruhenden Uhren sind synchronisiert, so dass
gleichzeitig von zwei Uhren ausgesendet Lichtsignale gleichzeitig in
der Mitte der Verbindungslinie ankommen.]

\begin{verbatim}
    ct
     |               "Raumzeit-Diagramm"
     |
ct_E +----------------o E
     |                |
     |                |
     |                |
     |                |
     |                |
     |                |
     +----------------+-------- x
                     x_E
\end{verbatim}


Koordinaten eines Ereignisses $E$: $x_E,y_E,z_E,ct_E$.

Ein Teilchen (Massenpunkt, Photon [Lichtsignal], Beobachter)
beschreibt eine ``Weltlinie'' in der Raumzeit.



\begin{verbatim}
  ct |        A                    B
     |        |                    /-
     |        |                  /-
     |        |                --
     |        |                 |
     |        |                 \
     |        |                 /\
     |        |                /
     |        |               |
     |        |               |
     |        |                -\
     |                           --
     +------------------------------------------   x
              x_0
\end{verbatim}

$A$ Weltlinie eines bei $x_0$ ruhenden Teilchens. $B$ Weltlinie eines
sich bewegenden Teilchens.

Weltlinie eines kr\"aftefreien Massenpunktes: Gerade (1. Axiom gilt in
SRT unver\"andert!)

Anstieg der Weltlinie $(ct)_{,x}=c t_{,x} = c/V_x$.

F\"ur Lichtgeschwindigkeit in x-Richtung $(ct)_{,x}=1$ (Anstiegswinkel
$45{}^\circ$).







\end{document}